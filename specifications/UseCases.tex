\documentclass[a4paper,10pt]{report}
\usepackage{fullpage}
\usepackage[british]{babel}
\usepackage[T1]{fontenc}
\usepackage{amsmath}
\usepackage{amssymb}
\usepackage[T1]{fontenc}
\usepackage[latin1]{inputenc} 
%\usepackage{amsthm} \newtheorem{theorem}{Theorem}
\usepackage{color}
\usepackage{float}



\usepackage{caption}
\DeclareCaptionFont{white}{\color{white}}
\DeclareCaptionFormat{listing}{\colorbox{gray}{\parbox{\textwidth}{#1#2#3}}}
%\captionsetup[lstlisting]{format=listing,labelfont=white,textfont=white}


\usepackage{alltt}
\usepackage{listings}
 \usepackage{aeguill} 
\usepackage{dsfont}
%\usepackage{algorithm}
%\usepackage{algorithmicx}
\usepackage{subfig}
\lstset{% parameters for all code listings
	language=Python,
	frame=single,
	basicstyle=\small,  % nothing smaller than \footnotesize, please
	tabsize=2,
	numbers=left,
%	framexleftmargin=2em,  % extend frame to include line numbers
	%xrightmargin=2em,  % extra space to fit 79 characters
	breaklines=true,
	breakatwhitespace=true,
	prebreak={/},
	captionpos=b,
	columns=fullflexible,
	escapeinside={\#*}{\^^M}
}


% Alter some LaTeX defaults for better treatment of figures:
    % See p.105 of "TeX Unbound" for suggested values.
    % See pp. 199-200 of Lamport's "LaTeX" book for details.
    %   General parameters, for ALL pages:
    \renewcommand{\topfraction}{0.9}	% max fraction of floats at top
    \renewcommand{\bottomfraction}{0.8}	% max fraction of floats at bottom
    %   Parameters for TEXT pages (not float pages):
    \setcounter{topnumber}{2}
    \setcounter{bottomnumber}{2}
    \setcounter{totalnumber}{4}     % 2 may work better
    \setcounter{dbltopnumber}{2}    % for 2-column pages
    \renewcommand{\dbltopfraction}{0.9}	% fit big float above 2-col. text
    \renewcommand{\textfraction}{0.07}	% allow minimal text w. figs
    %   Parameters for FLOAT pages (not text pages):
    \renewcommand{\floatpagefraction}{0.7}	% require fuller float pages
	% N.B.: floatpagefraction MUST be less than topfraction !!
    \renewcommand{\dblfloatpagefraction}{0.7}	% require fuller float pages

	% remember to use [htp] or [htpb] for placement


\usepackage{fancyvrb}
%\DefineVerbatimEnvironment{code}{Verbatim}{fontsize=\small}
%\DefineVerbatimEnvironment{example}{Verbatim}{fontsize=\small}

\usepackage{url}
\urldef{\mailsa}\path|sharyari@gmail.com |    
\newcommand{\keywords}[1]{\par\addvspace\baselineskip
\noindent\keywordname\enspace\ignorespaces#1}


\usepackage{tikz} \usetikzlibrary{trees}
\usepackage{hyperref}  % should always be the last package

% useful colours (use sparingly!):
\newcommand{\blue}[1]{{\color{blue}#1}}
\newcommand{\green}[1]{{\color{green}#1}}
\newcommand{\red}[1]{{\color{red}#1}}

% useful wrappers for algorithmic/Python notation:
\newcommand{\length}[1]{\text{len}(#1)}
\newcommand{\twodots}{\mathinner{\ldotp\ldotp}}  % taken from clrscode3e.sty
\newcommand{\Oh}[1]{\mathcal{O}\left(#1\right)}

% useful (wrappers for) math symbols:
\newcommand{\Cardinality}[1]{\left\lvert#1\right\rvert}
%\newcommand{\Cardinality}[1]{\##1}
\newcommand{\Ceiling}[1]{\left\lceil#1\right\rceil}
\newcommand{\Floor}[1]{\left\lfloor#1\right\rfloor}
\newcommand{\Iff}{\Leftrightarrow}
\newcommand{\Implies}{\Rightarrow}
\newcommand{\Intersect}{\cap}
\newcommand{\Sequence}[1]{\left[#1\right]}
\newcommand{\Set}[1]{\left\{#1\right\}}
\newcommand{\SetComp}[2]{\Set{#1\SuchThat#2}}
\newcommand{\SuchThat}{\mid}
\newcommand{\Tuple}[1]{\langle#1\rangle}
\newcommand{\Union}{\cup}
\usetikzlibrary{positioning,shapes,shadows,arrows}

\pagestyle{empty}

\title{\textbf{DragonQuest - Use Cases}}

\author{Jonathan Sharyari \and Sven Lundgren \and Kristian Johansson \and Bj{\"o}rn Forsberg}

\begin{document}
\maketitle



\section*{Use Cases - subdivision TODO}

\begin{table}[h!]
\label{template}
\caption{Use-Case template}
\begin{tabular}{|c| p{9cm}|c}
\hline
Use Case & Number ID to represent your use case & \\
Application & What system or application does this pertain to & \\
Name & The name of your use case, keep it short and sweet &  \\
Description	& Elaborate more on the name, in paragraph form. & \\
Primary Actor & Who is the main actor that this use case represents & \\
Precondition &	What preconditions must be met before this use case can start & \\
Trigger	& What event triggers this use case & \\ \hline
Basic Flow	& The basic flow should be the events of the use case when everything is perfect; there are no errors, no exceptions. This is the "happy day scenario". The exceptions will be handled in the "Alternate Flows" section. & Included in design? \\ \hline
Alternate Flows	& The most significant alternatives and exceptions & Included in design? \\
\hline
\end{tabular}
\end{table}

\begin{table}
\label{connection}
\caption{Game connection}
\begin{tabular}{|c| p{9cm}|c}
\hline
Application	& Client  & \\
Name & Game connection  & \\
Description	& The user has launched the game, and will choose between a set of available options (e.g. connect, settings, statistics, exit)  & \\
Primary Actor & Player & \\
Precondition &None & \\
Trigger & Running the game executable, or quitting a current game which is interpreted as a reinitialization comparable to restarting the executable.  & \\ \hline
Basic Flow & The player presses a connection button, upon which he/she is  requested to specify the IP-number of an active game (server) to connect to.  & NO \\ \hline
Alternate Flows & The user changes his/her mind and presses the exit button. The user is then asked to verify that he/she really want to end the game before the game is ended.  & NO \\
\hline
\end{tabular}
\end{table}


\begin{table}
\label{choosehero}
\caption{Choose hero and magic ring}
\begin{tabular}{|c| p{9cm}|c}
\hline
Application & Server & \\
Name & Choosing the hero and magic ring & \\
Description & The player chooses a Hero, depending on his turn of action. & \\
Primary Actor & Multiple players & \\
Precondition & At least 2 players in an active game. & \\
Trigger & The game is begun & \\ \hline
Basic flow & The users roll a die each. The action turns of the players is decided by the outcome of the dice (highest die-first to act=first action turn). The players then choose a Hero from the available sets of heroes in the order of their actions (when a Hero is chosen by a player, that hero is no longer available to other players). When the heroes have been chosen, then players choose a magic ring among the available magic rings in the reverse action turn order. & YES \\ \hline
Alternate Flows & Several players roll the same die, in this case, those players may roll their die again in order to break the tie. The other players are not affected, and their action turn relative to these players is based on the first roll of the die. (i.e., A rolls 4. B rolls 1. and C rolls 1. C and B roll again, resulting in 6 and 5. The order is now A, C, B and not C, B, A). & NO\\
\hline
\end{tabular}
\end{table}


\begin{table}
\caption{Make a regular move}
\label{move}
\begin{tabular}{|c| p{9cm}|c}
\hline
Application & Server & \\
Name & Make a move & \\
Description & The player chooses an adjacent, legal, square to move to. & \\
Primary Actor & Player & \\
Precondition & The player's turn. The rules allow the player to move. & \\
Trigger & The player chooses to move from his/her current square. & \\ \hline
Basic flow & The player moves to a selected square. A random tile is placed on the square provided that there is no tile in that square. The player is dealt a room card, the effect of which differs. See use case \ref{ambushopponent} TODO TODO & YES \\ \hline
Alternate Flow & If a new room tile is placed and that room tile contains a trap, the player is dealt a trap card instead of a room card. See use case TODO.  & NO \\\hline
\hline
\end{tabular}
\end{table}


\begin{table}
\caption{Ambush by an opponent}
\label{ambushopponent}
\begin{tabular}{|c| p{9cm}|c}
\hline
Application & Server & \\
Name & Ambush (goblin, orc, cave troll, skeleton or two orcs) & \\
Description & The player is ambushed, and must fight or flee. & \\
Primary Actor & Two players & \\
Precondition & None & \\
Trigger & The player is dealt a a room card specifying an ambush, b draws a coffin card specifying an ambush & \\ \hline
Basic flow & The player must choose whether wishes to fight, or attempt to escape. If the player chooses to fight, see use case \ref{fightopponent}. If the player attempts to escape, see use case \ref{escape} & NO\\
\hline
\end{tabular}
\end{table}


\begin{table}
\caption{Escaping an opponent}
\label{escape}
\begin{tabular}{|c| p{9cm}|c}
\hline
Application & Server & \\
Name & Flee & \\
Description & The player tries to flee from an opponent. & \\
Primary actor & Player whose turn it is & \\
Secondary actor & Player with previous turn. & \\
Precondition & none & \\
Trigger & The player tries to escape an opponent. & \\ \hline
Basic flow & Both player and opponent draw a power card randomly. If the player's power card has an escape value higher or equal to the monster's escape value, the attempt was successful and the user returns to the room he/she came from. The monster remains in the room. 

If the player's escape value is lower than the escape value of the opponent, the player will take damage equal to the damage value of the monster's power card. The player must then fight the opponent, see \ref{fightopponent}.
& NO \\

\hline
\end{tabular}
\end{table}



\begin{table}
\caption{Fighting an opponent}
\label{fightopponent}
\begin{tabular}{|c| p{9cm}|c}
\hline
Application & Server & \\
Name & Fight (goblin, orc, cave troll, skeleton or two orcs) & \\
Description & The player fights an opponent, represented by another player in the game (or AI). & \\
Primary actor & Player whose turn it is & \\
Secondary actor & Player with previous turn. & \\
Precondition & None & \\
Trigger & A fight is initiated according to \ref{ambushopponent}. & \\ \hline
Basic flow & During the first round both players draw a random Power card (if the player unsuccessfully tried to flee, the power card used in the attempt can not be chosen) and add it to their hand. Both players draw cards from the combat deck, adding up to a total of five cards on their hands, without showing them to the other player. Note that each player has only one power card in each battle, and this card is to be regarded as a combat card. Each hero and monster have individual power cards.

Each turn in the battle goes as follows:

Both players choose a combat and place them infront of themselves. The cards are displayed when both players have selected their cards. If one or both cards are a power card, see Use case \ref{fight_powercard}.

If the attack value of a player's Combat card is equal to or lower than the opponents attack value and his card has a counterattack icon matching the opponents attack type, the player may make a counterattack, see use case \ref{fight_counterattack}.

The player with the highest total attack value wins the round (possibly due to the effects of a counterattack). The loser has to put all the winners attack cards into their damage stack, and put their own cards in the combat stack.

If the winner played a combat card of a type that is currently in the combat stack, additional damage is dealt according to \ref{fight_deathblow}.

If both players have the same attack value and no counterattacks are possible, no player wins and all cards are placed in the combat stack. This is called a Stand-off.

This is repeated, until either the opponent or the player has received damage higher than or equal to the available HP. In this case, the fight ends and game continues.

If the opponent was the winner, the player is out of the gamem and the monster continues to occupy the room, without taking any damage. If the player was the winner, the monster is dead and the player will take damage equal to the number of cards in the damage stack.

& NO \\



\hline
\end{tabular}
\end{table}

\begin{table}
\caption{Counter-attack}
\label{fight_counterattack}
\begin{tabular}{|c| p{9cm}|c}
\hline
Application & Server & \\
Name & Counterattack & \\
Description & A counterattack is made by the player (or AI), during a battle. & \\
Primary actor & Player whose turn it is & \\
Secondary actor & Player with previous turn.& \\
Precondition & Player (monster) has used a combat card with lower value, but has a counterattack chance matching the attack type of the monster (player)& \\
Trigger & The player (monster) chooses to counterattack  & \\ \hline
Basic flow & The player is allowed to increase his/her initial attack value, by adding more cards to his attack pile. Only cards with a counterattack symbol matching the opponents attack may be put down. The player may keep adding cards to the attack, until the total sum of the attack values exceeds the attack value of the opponent.& NO\\
\hline
\end{tabular}
\end{table}


\begin{table}
\caption{Death-blow}
\label{fight_deathblow}
\begin{tabular}{|c| p{9cm}|c}
\hline
Application & Server & \\
Name & Death blow & \\
Description & A death blow adds additional damage to the opponent during battle & \\
Primary actor & Player whose turn it is & \\
Secondary actor & Player with previous turn. & \\
Precondition & The played attack card type matches the attack type of one or more cards in the combat stack, and the card has a higher attack value than the opponent & \\
Trigger &  & \\ \hline
Basic flow & The winning player may take all cards of the same type as the played card from the combat stack, and put them in the damage pile of the opponent & NO\\
\hline
\end{tabular}
\end{table}

% Combat discrepancy: Does a death blow happen before or after the counterattack? If counterattack happens first, the precondition of the death blow is no longer satisfied. We have chosen to regard death blow as a part of the damage assignment, i.e. after the counterattack


\begin{table}
\caption{Power card effects}
\label{fight_powercard}
\begin{tabular}{|c| p{9cm}|c}
\hline
Application & Server & \\
Name & Fight (goblin, orc, cave troll, skeleton or two orcs) & \\
Description & The player fights an opponent, represented by another player in the game (or AI). & \\
Primary actor & Player whose turn it is & \\
Secondary actor & Player with previous turn. & \\
Precondition & None & \\
Trigger & A fight is initiated according to \ref{ambushopponent}. & \\ \hline
Basic flow & The player, and the opponent both choose one of their three attack cards. When both have chosen, these are compared and the winner is determined. The looser takes one damage (the opponent can potentially take two damage). In case of a tie, both player and opponent take one damage. This repeats, until either the opponent or the player is dead. & NO\\ \hline
alternative flow & Crossbow TODO & YES\\
\hline
\end{tabular}
\end{table}











\begin{table}
\caption{Ambush by a monster}
\label{ambushmonster}
\begin{tabular}{|c| p{9cm}|c}
\hline
Application &	Server & \\
Name & Ambush (Monster, e.g. Great spider)& \\
Description&  The player is dealt a Room Card with an ambush by a Great Spider, and fights it.& \\
Primary actor & Active Player& \\
Precondition &  None.& \\
Trigger & The player is dealt a Room card/room searching card that has Great Spider ambush on it.& \\ \hline
Basic flow & The player has to choose three number between 1 and 6. The computer randomizes a number between 1 and 6, and if the generated number matches a number given by the user, the Great Spider dies and the turn is over. & NO \\ \hline
Alternate flow & If the generated number does not match one of the user given numbers the user is dealt 1 (one) damage point and the spider remains on the users hand until next turn, at which point the use case is restarted. & NO \\
\hline
\end{tabular}
\end{table}


\begin{table}
\caption{Occupying the dragon's lair}
\label{dragonslair}
\begin{tabular}{|c| p{9cm}|c}
\hline
Application &  Server & \\
Name &  Occupying the dragon's lair & \\
Description & The player is in the dragons lair and tries to steal gold from the sleeping dragon. & \\
Primary actor&  Player whose turn it is & \\
Precondition & None & \\
Trigger & The user enters, or remains in the dragons lair (treasure chamber). & \\ \hline
Basic flow & The user is dealt a random amounts of gold, and the computer decides if the dragon wakes up or not. If the dragon is not woken, the turn is over. & \\ \hline
Alternative flow & If the dragon wakes, all Players in the treasure chamber are evicted into a random adjacent and previously visited room and loose all gold they have gained since entering the treasure chamber. All players are also dealt a random value between 1-12 in damage. The turn is then over. & NO \\
\hline
\end{tabular}
\end{table}


\begin{table}
\caption{Search a room}
\label{searchroom}
\begin{tabular}{|c| p{9cm}|c}
\hline
Application & Server & \\
Name & Search room & \\
Description &  & \\
Primary actor & Active player & \\
Precondition & Active player have searched the room 0 or 1 time before. Room rules allows the room to be searched  & \\
Trigger & The active player chooses to search the room and draw a room searching card  & \\ \hline
Basic flow & The player draws a room searching card and finds nothing. The turn is over and number of searches is incremented. & \\ \hline
Alternative flow & The player draws a card with a hidden door and may or may not go through it. The hidden door may be "placed" in any direction and disappears after the turn. If the player does not go through the door the turn is over. & NO \\\hline
Alternative flow & A giant spider is drawn, see use case \ref{ambushmonster}. & YES \\ \hline
Alternative flow & A bottle is found & YES \\ \hline
Alternative flow & A treasure is found, player receives random amount of treasure. & YES \\
\hline
\end{tabular}
\end{table}

Application : Server
Name: Search coffin or fallen warrior (skriv bada gemensamt, tva olika basic flows)

Application: Server
Name: Open door

Application: Server
Name: Player dies


Server - await 3-4 players



\end{document}

