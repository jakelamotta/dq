\documentclass[a4paper,10pt]{article}
\usepackage{fullpage}
\usepackage[british]{babel}
\usepackage[T1]{fontenc}
\usepackage{amsmath}
\usepackage{amssymb}
\usepackage[T1]{fontenc}
\usepackage[latin1]{inputenc} 
%\usepackage{amsthm} \newtheorem{theorem}{Theorem}
\usepackage{color}
\usepackage{float}



\usepackage{caption}
\DeclareCaptionFont{white}{\color{white}}
\DeclareCaptionFormat{listing}{\colorbox{gray}{\parbox{\textwidth}{#1#2#3}}}
%\captionsetup[lstlisting]{format=listing,labelfont=white,textfont=white}


\usepackage{alltt}
\usepackage{listings}
 \usepackage{aeguill} 
\usepackage{dsfont}
%\usepackage{algorithm}
%\usepackage{algorithmicx}
\usepackage{subfig}
\lstset{% parameters for all code listings
	language=Python,
	frame=single,
	basicstyle=\small,  % nothing smaller than \footnotesize, please
	tabsize=2,
	numbers=left,
%	framexleftmargin=2em,  % extend frame to include line numbers
	%xrightmargin=2em,  % extra space to fit 79 characters
	breaklines=true,
	breakatwhitespace=true,
	prebreak={/},
	captionpos=b,
	columns=fullflexible,
	escapeinside={\#*}{\^^M}
}


% Alter some LaTeX defaults for better treatment of figures:
    % See p.105 of "TeX Unbound" for suggested values.
    % See pp. 199-200 of Lamport's "LaTeX" book for details.
    %   General parameters, for ALL pages:
    \renewcommand{\topfraction}{0.9}	% max fraction of floats at top
    \renewcommand{\bottomfraction}{0.8}	% max fraction of floats at bottom
    %   Parameters for TEXT pages (not float pages):
    \setcounter{topnumber}{2}
    \setcounter{bottomnumber}{2}
    \setcounter{totalnumber}{4}     % 2 may work better
    \setcounter{dbltopnumber}{2}    % for 2-column pages
    \renewcommand{\dbltopfraction}{0.9}	% fit big float above 2-col. text
    \renewcommand{\textfraction}{0.07}	% allow minimal text w. figs
    %   Parameters for FLOAT pages (not text pages):
    \renewcommand{\floatpagefraction}{0.7}	% require fuller float pages
	% N.B.: floatpagefraction MUST be less than topfraction !!
    \renewcommand{\dblfloatpagefraction}{0.7}	% require fuller float pages

	% remember to use [htp] or [htpb] for placement


\usepackage{fancyvrb}
%\DefineVerbatimEnvironment{code}{Verbatim}{fontsize=\small}
%\DefineVerbatimEnvironment{example}{Verbatim}{fontsize=\small}

\usepackage{url}
\urldef{\mailsa}\path|sharyari@gmail.com |    
\newcommand{\keywords}[1]{\par\addvspace\baselineskip
\noindent\keywordname\enspace\ignorespaces#1}


\usepackage{tikz} \usetikzlibrary{trees}
\usepackage{hyperref}  % should always be the last package

% useful colours (use sparingly!):
\newcommand{\blue}[1]{{\color{blue}#1}}
\newcommand{\green}[1]{{\color{green}#1}}
\newcommand{\red}[1]{{\color{red}#1}}

% useful wrappers for algorithmic/Python notation:
\newcommand{\length}[1]{\text{len}(#1)}
\newcommand{\twodots}{\mathinner{\ldotp\ldotp}}  % taken from clrscode3e.sty
\newcommand{\Oh}[1]{\mathcal{O}\left(#1\right)}

% useful (wrappers for) math symbols:
\newcommand{\Cardinality}[1]{\left\lvert#1\right\rvert}
%\newcommand{\Cardinality}[1]{\##1}
\newcommand{\Ceiling}[1]{\left\lceil#1\right\rceil}
\newcommand{\Floor}[1]{\left\lfloor#1\right\rfloor}
\newcommand{\Iff}{\Leftrightarrow}
\newcommand{\Implies}{\Rightarrow}
\newcommand{\Intersect}{\cap}
\newcommand{\Sequence}[1]{\left[#1\right]}
\newcommand{\Set}[1]{\left\{#1\right\}}
\newcommand{\SetComp}[2]{\Set{#1\SuchThat#2}}
\newcommand{\SuchThat}{\mid}
\newcommand{\Tuple}[1]{\langle#1\rangle}
\newcommand{\Union}{\cup}
\usetikzlibrary{positioning,shapes,shadows,arrows}

% SRS commands
\newcommand{\requirement}[1]{\subsubsection{#1}\begin{tabular}{l p{12.2cm}}}

\newcommand{\reqsection}[1]{\\ \textbf{#1} &}

\newcommand{\stoprequirement}{\end{tabular}}

\newcommand{\prio}[2]{\texttt{Priority:} #1 \emph{Motivation:} #2}


\pagestyle{empty}

\title{\textbf{DragonQuest - Software Requirement Specification}}

\author{Jonathan Sharyari \and Sven Lundgren \and Kristian Johansson \and Bj{\"o}rn Forsberg}

\begin{document}
\maketitle

This document contains the requirements for the DragonQuest game. All requirements have a priority associated with them, ranging from $1$ to $5$, where $1$ is the highest and $5$ is the lowest. This implies that if some functionality is inconsistent or for some other reason is not realizable, the priorities reflect which requirements have precedence to be implemented in the software.

\section{External Interface Requirements}
\subsection{User Interfaces}

\requirement{Chat}
\reqsection{Description}
Interface requirement: There should be a chat window with two fields, one for input and one to display recent messages. Both fields should have support for all languages based on the latin and cyrillic alphabet, that are needed to communicate in those languages.

\reqsection{Inputs}
Inputs can be keyboard, mouse and touchpad.

\reqsection{Processing}
Pressing the text field with the mouse or the touchpad should change the focus to the text field, in order to allow text input from the keyboard. From this point on, the user can type text, which will be submitted when the user presses the enter key. 

\reqsection{Output}
The chat display field displays the 5 latest lines of text. 

\reqsection{Error handling}
If input character doesn't exist that fact should not affect the rest of the provided characters. If data is lost between clients it should be re-sent.

\reqsection{Priority}
5

\stoprequirement

\subsection{Hardware Interfaces}
\subsection{Software Interfaces}
\subsection{Communications Interfaces}
\section{Functional Requirements}
This section describes specific features of the software project.  If desired, some requirements may be specified in the use-case format and listed in the Use Cases Section.

\subsection{Game settings}

\requirement{Multiplayer/Single Player Mode}
\reqsection{Description}
The user should be able to choose to play the game in multiplayer format, or single player format.

\reqsection{Inputs}
The user should either provide an IP-address, as a dot-separated number or a hostname to connect to or alternatively, the user should be able to create a new game.

\reqsection{Processing}
The client will try to establish a connection to an external server on the host with the supplied IP-address. 
If this does succeed, the user will enter the existing game.

If the supplied IP-address is localhost, and no server is running on the same computer as the client, a new game will be started. When a new game is started, the user should be able to decide game parameters, most importantly the number of allowed external players and the number of players in total.

Note that the number of total players may be higher than the number of external players, i.e., the remaining players are controlled by the server AI. Creating a single player game then corresponds to the act of not allowing any external players.

\reqsection{Side-effect}
A game starts.

\reqsection{Error handling}
If game connection does not succeed, an error message is shown to the user describing the reason for the failure. The most common reasons of failure should be recognized, this includes hostname not being found, game not being found, game full and game has already started. 

\reqsection{Priority}
1 \emph{Motivation:} Business value, architecturally important, critical for game functionality

\stoprequirement

\requirement{Observing game}

\reqsection{Description}
A user not involved in an ongoing game can choose to observe any game via its client. 

\reqsection{Inputs}
The user provides to it's client the IP-address of the server hosting the game he/she want's to observe.

\reqsection{Processing}
An attempt will be made to set up the connection and if successful the user will join the game and current game state will be loaded to the client.

\reqsection{Output}
The game state is loaded and the client presents the information to the user according to its implementation.

\reqsection{Error handling}
If the connection is not successfull the user will be presented with information that so was the case and why it didnt work. After the user will have the option to retry the connection attempt. 

\reqsection{Priority}
3 \emph{Motivation:} Additional feature that allows non-players to watch games, which tightens the community around the game.

\stoprequirement

\subsection{Game board actions}

\requirement{Move}

\reqsection{Description}
The user can move its hero to another room or into a tower.

\reqsection{Inputs}
The user should provide a move command to the server, depending on how the client is designed how this is done might vary.

\reqsection{Processing}
Except for validation (validation is also done on the server side) the client doesn't do any processing of the input. The input is passed to the server where its dealt with, a move will in normal cases be to any adjacent room (or up or down a staircase) that is allowed. In certain rooms the user has the option to move two or potentially more steps as well but these rules are set by the specific room.

\reqsection{Side-effects}
The players turn is over.

\reqsection{Output}
The hero in question has been moved to the square the move aimed for.

\reqsection{Error Handling}
Any non-valid user inputs are dealt with by the client and a message that the move is invalid is displayed for the user.
Invalid data can still reach the server if the data is altered on the way. Therefore the server also validates the data to make sure the move is allowed according to the game state and rules.
If the client loses its connection to the server the user will get made aware that the command wasn't executed and that the action should be tried again.

\reqsection{Priority}
1 \emph{Motivation:} Critical for the game's functionality

\stoprequirement

\requirement{Search room}

\reqsection{Description}
The user can search a room for items such as chests, hidden doors, treasures etc.
 
\reqsection{Inputs}
The user gives the search room command via the client.
 
\reqsection{Processing}
The room search can start at any point given that it's the player's turn and it stands in a searchable room (not all rooms are).

When the search commences the player draws a room search card which will reveal what the player found and further action will take place accordingly.
 
\reqsection{Side-effect}
The card drawn will determine the outcome of the search. See requirements for the specific actions.
The players turn is over.
 
\reqsection{Error Handling}
Any non-valid user inputs are dealt with by the client and a message that the move is invalid is displayed for the user.
 
Invalid data can still reach the server if the data is altered on the way. Therefore the server also validates the data to make sure the move is allowed according to the game state and rules.
 
If the client loses its connection to the server the user will get made aware that the command wasn't executed and that the action should be tried again.

\reqsection{Priority}
3 \emph{Motivation:} Important for game experience

\stoprequirement

\requirement{Occupying the dragon's lair}

\reqsection{Description} A player currently residing in the dragon's lair can decide to stay there for another turn in risk of awakening the dragon and in return taking damage from it. 

\reqsection{Inputs} The user chooses to wait when occupying the dragon's lair.

\reqsection{Processing} When the user issues the "stay" command a dragon card is drawn and this will either show that the dragon stay asleep in which case no further action is taken. But the result can also be that the dragon is awaken in which case the hero will receive a random amount of damage and also kicked out of the lair to an adjacent tile.

\reqsection{Output} Either nothing (dragon stays asleep) or damage suffered and movement of the player's hero.

\reqsection{Error handling} If the player loses connection and cant reconnect in time an AI will take over and play the turn for him/her. 

\reqsection{Priority}
2 \emph{Motivation:} The main goal of the game from the player's perspective

\stoprequirement

\subsection{Battling}

\requirement{Battle}

\reqsection{Description}
A player can battle an opponent, controlled either by another player or the AI.

\reqsection{Inputs}
The player and opponent both choose a card from their respective battle hands. The player uses a mouse or touchpad to do so.

\reqsection{Processing}
The winner of the battle is determined by comparing the two cards. The participant with the highest total attack value wins the round, and the opponent loses one as many hit points as the number of cards the winner of the round played. There is a possibility of deathblow and counter attack, according to their respective requirements.

\reqsection{Output}
The player and opponent are visually informed of the outcome of the battle round.

\reqsection{Priority}
1 \emph{Motivation:} Important for game experience, architecturally important

\stoprequirement

\requirement{Ambush}

\reqsection{Description}
A player can be the victim of an ambush by an opponent when entering a new room.

\reqsection{Inputs}
The player may choose whether to fight the opponent or flee from the battle. The choice is made using a mouse or touchpad.

\reqsection{Processing}
If the player chooses to fight, a battle between the player and opponent commences. If the player chooses to flee, the player and opponent compares their respective power cards. If the player's power card has a higher attack value, the player successfully flees. Otherwise, the player takes damage ??? and a battle between the player and opponent commences.

\reqsection{Output}
If the player chooses to flee, the player is informed visually of the outcome.

\reqsection{Error handling}
See Requirement \ref{req:connectionlossstall}.

\reqsection{Priority}
3 \emph{Motivation:} Important for board game correlation, \ref{req:boardgamecorrelation}

\stoprequirement

\requirement{Death blow}

\reqsection{Description} A player will do a death blow to an opponent thus doing extra damage to the opponent during that combat turn if the played attack card is matching any card in the combat stack and the value of the card is higher than that of the opponent.

\reqsection{Inputs} The attack card played by the player in question

\reqsection{Processing} The type of the attack card played is compared to the type of the cards in the combat stack. If the the type matches and the value of the player's card is higher the player may take all cards of that type from the combat stack and put it into the opponents damage stack. The battle turn is than resolved as usual.

\reqsection{Output} The output is the changed amount of damage the opponent takes as a result of the successful death blow

\reqsection{Error handling} If a player disconnects and does not re-establish connection within the specified round time, AI will take over until the player returns.

\reqsection{Priority}
5 \emph{Motivation:} Not critical for game experience.

\stoprequirement

\requirement{Counter Attack}

\reqsection{Description} A player may counter attack after the initial battle cards have been played.

\reqsection{Inputs} Provided that the player has a counter attack card that matches the type of the opponents chosen attack a counter attack can be made by providing one or more CA cards. 

\reqsection{Processing} The CA cards that are added to combat stack will increase the total attack value of the player in question. After these cards have been played the battle round continues as usual.

\reqsection{Output} The attack value of the counter attacking player will be increased accordingly.

\reqsection{Error handling} If a player disconnects and does not re-establish connection within the specified round time, AI will take over until the player returns.

\reqsection{Priority}
4 \emph{Motivation:} Adds a competitive factor to the game, see \ref{req:competition}


\stoprequirement

\subsection{Connections}

\requirement{Connection loss/stall}
\label{req:connectionlossstall}
\reqsection{Description}
The client should be able to reconnect to the game server without losing the game state.

\reqsection{Inputs}
N/A

\reqsection{Processing}
If the client reconnects within the servers specified turn time the game continues. If the client is unable to reconnect one of the following happens:
\begin{itemize} 
	\item If player not in battle: The players turn is forfeit and the player loses 1 HP.  
	\item If player in battle: AI takes over until the player returns.
\end{itemize}

\reqsection{Priority}
1 \emph{Motivation:} Critical for playability and business.


\stoprequirement

\section{Non-Functional Requirements}
Non-functional requirements may exist for the following attributes.  Often these requirements must be achieved at a system-wide level rather than at a unit level.  State the requirements in the following sections in measurable terms (e.g., 95\% of transaction shall be processed in less than a second, system downtime may not exceed 1 minute per day, > 30 day MTBF value, etc). 
\subsection{Performance}

\subsection{Game experience}
\subsubsection{Competition}
\label{req:competition}
The game contains elements that allow for competativeness between players.

\prio{3}{Replay value, which adds business value.}

\subsubsection{Game community}
The should exist a community within the game to keep the players as engaged as possible. This includes the abilities to keep track on other players (friends), the ability to change the game environment (on the client side) to the largest extent possible, and the ability to create and share user generated content within the community.

\prio{3}{Community creates business opportunities around the game.}

\subsubsection{Medievalness}
The game should have provide a medieval experience, including music, graphics and language.

\prio{3}{Users expect this from a game with a medieval setting [quotation needed].}

\subsubsection{Board game correlation}
\label{req:boardgamecorrelation}
The game should deviate as little as possible from the standard board game, unless otherwise specified in this document.

\prio{2}{Business value, as selling the game as a computer game copy of the board game eases marketing.}

\subsection{Reliability}
\subsubsection{Responsiveness}
The game should stall as little as possible in order to improve the game experience. The game should also minimize stalls caused by users.

\prio{1}{Critical to keep users playing.}

\subsubsection{Stability}
The game shall not respond unexpectedly to erronous user input.

\prio{1}{Critical for game playability.}

\subsection{Availability}
\subsubsection{Learning curve}
It should be possible to learn the game adequately within 20 minutes of game play. No tutorial should be needed.

\prio{3}{A shallow learning curve eases recruitment of new players.}

\subsubsection{Gameplay development}
The game should include elements of gameplay that allow the user to develop as a player. Also, the characters in game should evolve in some extent based on in-game experiences, in order to keep the game as versatile as possible.

\prio{3}{Increases probability of players keep playing, which gives market opportunities}

\subsection{Security}
\subsubsection{Validation}
\label{cheating}
The server should validate the client input to make sure they are legal according to the game rules. The client needs to validate server data user input. 

\prio{1}{Multiplayer cheating is a common cause of player loss [citation needed]}

\subsubsection{Privacy}
Any collected informations should be handled in accordance with Personuppgiftslagen (PUL), the swedish personal data law.

\prio{1}{This is required in order to follow Swedish law.}

\subsection{Maintainability}
\subsubsection{Maintenance}
The code should be easy to maintain and extend.

\prio{1}{The game is expected to be used for a long time after the initial release. It is essential to keep maintenance low.}


\subsubsection{Reusability}
The code base of the project should be as little intertwined as possible with the specific game rules of DungeonQuest. This should allow for code reuse, when developing other board games in the future.

\prio{1}{Code re-usability is important, as we expect to develop a multitude of DragonQuest-based games in the future.}

\subsection{Portability}
\subsubsection{Mobile devices}
The game should support the most common mobile devices, such as the Windows mobile, iOS and Android platforms. It shall be possible to create a lower resolution client, in order to support any phone that fulfills the minimum requirements stated in \ref{hwreq}.

\prio{2}{The mobile operating systems are still young, and it is likely that some of these operating systems will be obsolete in the near future.}

\subsubsection{Multi-platform}
The game should be written in Java, in order to function on any platform supporting the java virtual machine. As iOS currently does not support the machine, a objective-C version of the \emph{client} should be made available. An iOS compatible version of the server will not be prioritized before the initial game release.

\prio{2}{Important to make the game available to as many players as possible, increasing the revenue.}

\section{Design Constraints}
Specify design constrains imposed by other standards, company policies, hardware limitation, etc. that will impact this software project.

\subsection{Process}
\subsubsection{Iterative development process}
The game development should follow an iterative design process. Each iteration shall be two weeks long, and a snapshot of the development shall be handed in for review to the customer at the end of each iteration.

\prio{1}{Iterative development processes are shown to produce significantly better results.}


\subsubsection{Deadlines}
The game must be ready for deployement by christmas 2013.

\prio{2}{Important for business, exploit christmas sales.}


\subsection{Product}
\subsubsection{Low hardware requirements}
\label{hwreq}
The hardware requirements of the game should be kept as low as possible. The game should run on any computer with at least 20mb RAM and 300mhz processor and a graphics card with 3d accelleration. (We're just making this up. It would be possible to have this kind of constraint but we don't know what numbers are reasonable).

\prio{2}{Important to make the game available to as many players as possible, increasing the revenue.}


\subsubsection{Data transfer}
The game should rely on low amounts of data transfer.

\prio{2}{Avoids bandwidth restrictions, allowing the game to be played with less stable internet connection.}


\subsubsection{Multi-client}
The game should have a server-client architecture. It should be possible for users to design 3rd party clients, but not 3rd party servers for the game. The server should ensure that all clients follow the game rules, also see \ref{cheating}.

\prio{3}{This is expected to strengthen the community surrounding the game.}


\end{document}

